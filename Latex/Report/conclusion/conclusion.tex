\chapter{Discussion}\label{ch:discussion}
%This chapter discusses the possible solution to some problems. Furthermore, ideas to what could be done further is given. \fxnote{mangler en intro i diskution}
%Der burde laves noget til at simulere kemiske reaktioner
%    WATS model
A basic study of how a WWTP works is performed, and from this a problem statement is formulated. A simulation environment is constructed which can simulate setups which can easily be expanded by adding pipes and tanks in a setup function. Limitations is put on the chemical reactions occurring in sewer lines partly because of the delimitations made but also because of the complexity of it. Further research could be made into the Wastewater of Aerobic/Anaerobic Transformation in Sewers (WATS) model, which specifically is made to simulate chemical reactions in sewer lines. Assumptions is made utilizing the Saint-Venant equations, which simplifies the flow in the sewer line greatly. Even though simplifications is made, numerical errors in the chosen solution can be utilized to obtain a wave which mimic the real world. By utilizing Courant's number the degree of numerical error can at some extent be known during simulation. It is seen that an accurate computation can be obtained if $\Delta t$ and $\Delta x$ is chosen precisely. But an accurate result does not represent a real world scenario well. instead some discrepancy is desired such that effects such as hydraulic jump in front of a wave going through the sewer line occurs.
Therefore a Courant's number close to one is preferable to obtain some realism when simulating. 

Another problem is that if $\Delta x$ is chosen small the computational task can increase considerably, if a large setup of pipes is simulated. Furthermore the increase in size causes severe problems when linearizing a model to be utilized with MPC. The reason for this is in part the chosen linearization method which causes the state space system to be ill conditioned, which gets worse when more sections is added. When predicting the state equations the problem is worsened, and causes numerical error which minimizes the prediction horizon considerable. Solutions could be to create simple model such that MPC could be used as a top level controller, or sample iterations of the state space system and do zero order hold at the MPC. But common for both is that prediction into the future is necessary, but not always available, depending on the location of the tank compared to where the disturbance is located.






%Due to numerical problems, MPC would not be able to be implemented on the simulation model for Fredericia, as the linearized model have too high dimensions, and because the linearized model consists of numerical small numbers. Therefore when predicting the future model in MPC problems arises, as these numerical small numbers will start to grow even smaller. However, if model order reduction could be implemented to lower the complexity of the large-scale system then there might be a possibility for to increase the prediction horizon and thereby allowing the MPC control to predict further into the future. 

%As stability and precision are an important parameter to consider when simulating, this Courant number needs further investigation. As it highly variates with the average flow height, time step and distance step it is hard to find an optimal $\Delta t$ and $\Delta x$ when dealing with pipes of different lengths. However, if every pipe could be split up into the same section lengths a more optimal Courant number could be obtained. An even better solution would be to make variable time step for each pipe, thereby it would be able to obtain the desired Courant number, which would make the simulation less affect by distortion caused by numerical errors. 

\chapter{Conclusion}\label{ch:conclusion}

The focus of this project was to create a simulation model that was able to mimic the behavior of a real sewer system. Furthermore, MPC were to be implemented to obtain a low variating input into the WWTP based on the problem statement from section \ref{sec:problem_statement}. 

\begin{center}
\textit{How can a simulation environment be constructed, which mimic the behavior of a real sewer system, where MPC is utilized as the control scheme to obtain stable sewage output such that optimal performance can be obtained from a WWTP.}
\end{center}


A simulation model has been designed where it is possible to create a sewer system consisting of several pipes, tanks and side inflows. It is able to simulate wastewater flow and concentrate throughout a city. Furthermore, it is constructed in a way, where it can be reconfigured to fit other setups by changing the order of pipes and tanks to obtain a desired simulation setup. 
The results show that even though a simplified numerical scheme was chosen to simulate flow in pipes, some realism could be attained by utilizing numerical errors.

%The results showed that it is able to simulate a city, where different flow profiles are constructed for the residential and industrial areas and the larger industries. This showed that it was able to produce an output amount that was similar to the one given by Fredericia. However, the simulation model seems to variate in the output more than the data given by Fredericia. The reason is, that in the simulation setup a pump before the WWTP has not been accounted for, and therefore the flow into the WWTP is more smooth in the real case. Furthermore, the flow profiles from the residential and industrial areas have been connected directly onto the main sewer line in the simulation, and therefore are these not as attenuated as they would be in a real life scenario, which causes higher peaks in the results.      

An MPC controller was constructed to control the output of a tank such that output variations were to be minimized. However, due to the rapid increase in size caused by the large number of sections numerical problems related to the linearized model appeared. Specifically the prediction horizon were found to be limited in size if the quadratic problem could be solved.
A controller were designed and tested but the results were less than satisfying.

%that it was not able to have a high prediction horizon as the quadratic problem became non-convex thereby having no solution to the problem. Therefore, the sewer system was reduced in size and a prediction horizon was found. Without constraints, the controller was able to keep a constant output, but at the cost of the tank was overfilled. Therefore constraints were introduced into the problem. However, this did not give the intended outcome. The controller started to produce an output that was not optimal, as it variated the output of the tank to a larger extent. Therefore, this controller needs further investigating in how to make it work properly with constraints.        

Even though the project is not in its current form finalized, advantages can be seen in the simulation model as a tool to implement counter measures against flow variations. 


 %is deemed as a plausible solution to simulate wastewater flow in urban environments and with more work into the MPC it can be used to analyze a real sewer network. Where it can be used to find the most optimal location to place tanks. Thereby reducing the variations in flow and concentration into the WWTP.  



