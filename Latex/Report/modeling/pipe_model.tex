
\subsection{Pipe model}\label{se:pipe_model}
When the wastewater requires to be elevated from lower geographical point to another it is done using a pump connected with pipes to lead the wastewater to a higher point. Therefore a model of a pipe is needed to describe the flow change within this pipe. Therefore in this subsection a model for a pipe will be elaborated. 
%To calculate the flow from the valve, seen in figure \ref{fig:block_diagram_system_overview}, the pressure losses in the Mickey Mouse setup must be known. This section will establish the equation for calculating the pressure loss across a pipe. 

In figure \ref{fig:pipe_3d} an illustration of a pipe is shown.
\begin{figure}[H]
\centering
\includegraphics[width=0.9\textwidth]{report/modeling/pictures/pipe_3d.pdf}
\caption{Illustrate the forces within a pipe.}
\label{fig:pipe_3d}
\end{figure}

Where the three forces shown in figure \ref{fig:pipe_3d}, $F_{in}$, $F_r$ and $F_{out}$. $F_{in}$ is the force going into the pipe, $F_r$ is the resistance force within the pipe, such as pipe wall and bends. And lastly $F_{out}$ which is the force going out of the pipe. These forces are effecting the wastewater inside the pipe and is equal to the rate of change in the momentum of the water. 

\begin{equation}
	\frac{d\mathcal{M}}{dt}=F_{sum}
\end{equation}

Where $\mathcal{M}$ is the momentum $\left[\frac{kg\cdot m}{s}\right]$ and $F_{sum}$ is the sum of all the forces $[N]$. Assuming that the water is incompressible and the pipe is filled with wastewater the following can be obtained: 

\begin{equation}\label{eq:pipe_newton}
	\frac{d\mathcal{M}}{dt}=\frac{d(M\cdot v)}{dt}=M\frac{dv}{dt}=F_{sum}
\end{equation}
%The derivation of the pipe model is based on figure \ref{fig:pipe_3d}, the pipe model is made, with three different forces, $F_{in}$ which is the pressure force going into the pipe, $F_r$ which is the resistance force in the pipe also known as drag force and lastly $F_{out}$ which represents the pressure force pushing to the left.
Where M is the mass of the water $[kg]$ and v is the velocity $\left[\frac{m}{s}\right]$.  
% Newton's second law applied for the pipe, 
% \begin{equation}\label{eq:pipe_newton}
% M\cdot \frac{d}{dt}v = F_{in} - F_{out} - F_r 
% \end{equation}
% Where $M$ is the mass of the water $[kg]$, $v$ is the velocity of the water $\left[\frac{m}{s}\right]$ and the resultant forces $F_{in}$, $F_r$ and $F_{out}$ $[N]$.  
The mass of the water can be written as density times the volume of the pipe,

\begin{equation}
M = \rho \cdot V 
\end{equation}

Where $\rho$ is the density of liquid $\left[\frac{kg}{m^3}\right]$ and $V$ the volume of the pipe $\left[m^3\right]$.
%The equation \ref{eq:pipe_newton} is expressing that the water mass multiplied with the water acceleration is equal to the resultant force.\\
%The mass of the water can be expressed as: 
%\begin{equation}
%m = \rho \cdot V \enhed{kg}
%\end{equation}
%Where:\\
%$\rho$ is the density of the water. $\enhed{kg/m^3}$\\
%$V$ is the volume of the pipe. $\enhed{m^3}$\\
Assuming the pipe is a cylinder, with a constant cross-sectional area along the pipe, the volume of the pipe can be written as, 
\begin{equation}
 V=A \cdot l
\end{equation}
Where A is the area of a circle $[m^2]$ and $l$ is the length of the pipe [m].
%$A$ is the area of a circle. $\enhed{m^2}$\\
%$l$ is the pipe length. $\enhed{m}$\\
Inserting the previous expressions in equation \ref{eq:pipe_newton} the following is obtained,
\begin{equation} \label{eq:pipe_newton_2}
\rho \cdot A \cdot l \cdot \frac{d}{dt}v = F_{in}-F_{out}-F_r 
\end{equation}

The velocity of the water is written as,
\begin{equation}
v=\frac{Q}{A} 
\end{equation}

Where $Q$ is the flow through the pipeline $\left[\frac{m^3}{s}\right]$.

The input, output and resistance force are written as $F=p \cdot A$, where $p$ is the pressure [Pa]. Inserting the expression in \ref{eq:pipe_newton_2} the following is obtained,
\begin{equation}
\rho \cdot A \cdot l \cdot \frac{d}{dt}\left(\frac{Q}{A}\right) = p_{in}\cdot A - p_{out}\cdot A - p_r \cdot A 
\end{equation}

Simplified to,
\begin{equation}\label{eq:pipe_w_pr}
\frac{\rho \cdot l}{A}\cdot \frac{d}{dt}Q = \Delta p - p_r
\end{equation}
Where the pressure difference is expressed as $\Delta p$ and the area is divided on both sides.

The resistance pressure can now be divided into two resistance, namely form resistance and surface resistance \cite{swamee_pipe}. 

\textbf{Surface resistance:}\\
The surface resistance describes the pressure loss across a straight pipe \cite{swamee_pipe}.
\begin{equation}\label{eq:surface_resistance}
h_f=\frac{8\cdot f\cdot l}{\pi^2\cdot g\cdot d^5} \cdot |Q|\cdot Q
\end{equation}
Where $h_f$ is the head loss for the surface resistance of the pipe $[m]$, $f$ is a coefficient for surface resistance, also knows as the friction factor, $g$ is the gravitational acceleration $\left[\frac{m}{s^2}\right]$ and $d$ is the diameter of a circular pipe $[m]$.

\textbf{Form resistance:}\\
The form resistance describes the pressure losses from fittings in a hydraulic network \cite{swamee_pipe}. 
\begin{equation} \label{eq:form_resistance}
h_l=k_L \frac{ |v|\cdot v }{2\cdot g}
\end{equation}
Where $h_l$ is the head loss for the form resistance of the pipe [m] and $k_L$ is the form-loss coefficient. The form-loss coefficient varies for different components e.g. threaded elbows and tees. The coefficient for the different components can be found in \cite{fundamentals_of_fluid_mechanics}. 

The pressure resistance in the pipe is expressed as:\fxnote{Det lød til, at tom gerne vil have det her til at ligne openchannel}
\begin{equation}\label{eq:pr_resistance}
p_r=\rho \cdot g \cdot h_n 
\end{equation}

By substituting $p_r$ in equation \ref{eq:pipe_w_pr} with equation \ref{eq:pr_resistance} the following is obtained,

\begin{equation} \label{eq:pipe_sub_pr}
\frac{\rho \cdot l}{A}\cdot \frac{d}{dt}Q = \Delta p - \rho \cdot g \cdot h_n 
\end{equation}
Where $h_n$ is the form and surface resistance. The surface and form, equations \ref{eq:surface_resistance} and \ref{eq:form_resistance}, can be inserted into equation \ref{eq:pipe_sub_pr},

% \begin{equation} \label{eq:pipe_sub_pr_2}
% \frac{\rho \cdot l}{A}\cdot \frac{d}{dt}Q = \Delta p - \rho \cdot g \cdot h_l - \rho \cdot g \cdot h_f
% \end{equation}

% Inserting the form resistance from equation \ref{eq:form_resistance} and the surface resistance from equation \ref{eq:surface_resistance} into equation \ref{eq:pipe_sub_pr_2} following equation is achieved,

\begin{equation}\label{eq:final_pipe}
\frac{\rho \cdot l}{A} \cdot \frac{d}{dt}Q = \Delta p - \rho\cdot g\left(k_L \frac{ |v|\cdot v }{2\cdot g}+ \frac{8\cdot f  \cdot l}{\pi^2\cdot g \cdot d^5} \cdot |Q| \cdot Q\right) 
\end{equation}

Substitute the head loss factors with $k_v$, a coefficient describing the pressure losses in the pipes, and $\frac {\rho\cdot l}{A}$ is inertance $J$. The final pipe model is expressed as,

\begin{equation}\label{eq:pipe_model}
  \boxed{J\cdot \frac{d}{dt}Q = \Delta p - \rho \cdot g \cdot k_v \cdot |Q| \cdot Q}
\end{equation}
