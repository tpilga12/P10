\section{Summary of company visits}\label{app:resume}
This appendix contains a summary from a meeting with the Fredericia wastewater department. The summary is in danish. 

Virksomheds besøg 19/03/2018 

Spildevand der kommer til rensning ved Fredericia rensninganlæg stammer hovedsageligt fra industrien, 60-65 \%. Ved Carlsberg og Arla er der flow målinger. Der er ikke målinger fra beboelse, hverken flow eller koncentrat, dog er der flow målinger ved nogle af pumpe stationer, samt flow og koncentrations målinger ind på rensningsanlægget. Stofmængden er ukendt fra det meste af industrien. Der er biotector ved Carlsberg (TOC) og COD måler ved Arla. Fredericia kunne evt. skaffe flowmålinger til os efter kontakt med industrien. 
På nuværende tidspunkt regulerer Carlsberg deres spildevand så det har en pH værdi mellem 6 og 9. Carlsberg har også et spare bassin. Arla har to spare bassiner, hvor de også kontrollere deres pH udledning. Shell har deres eget rensningsanlæg.
Industrien er typisk gode til at holde en konsistent udledning af flow og koncentrat, der kan dog forekomme uheld. Fordelene ved Fredericia er, at temperaturen på spildevandet i kloakkerne ligger omkring 16-17 grader året rundt. Dette hjælper bakterierne med denitrificering af spildevandet. Bakterierne er mindre aktive med nitrificeringen og denitrificering når temperaturen kommer under 10 grader celsius. Hvilket betyder, at fjernelsen af nitrogen går langsommere. 
\begin{itemize}
	\item Der er problemer i ledningsnettet når der falder kraftig regn, der kan forekommer overløb, derudover gør man rensningsprocessen hurtigere ved kun at føre vandet igennem den mekaniske rensning og derefter udlede det til Lillebælt. 
	\item Der er kul filter på næsten alt for at fjerne lugtgener, såsom dæksler, overtryksventiler og lukkede bassiner. 
	\item Man vil gerne minimere opholdstid i spare bassiner for at undgå produktion af hydrogen sulfid. 
	\item Ved vedligeholdelse af rensningsanlægget lukkes hovedledningen ind til rensningsanlægget, hvor det er muligt at stuve spildevand op i hoved ledning i 3-4 timer i tørvejr.
	\item Grundvandsindtrængning er forhøjet under regnvejr, samt forhøjet når vandstanden i Lillebælt er over normen.
	\begin{itemize}
		\item Forhøjet vandstand kan øge klorid indholdet i spildevandet både ved at trænge ind gennem grundvandet, men også ved tilbageløb i overløbsanlæg beliggende ud til Lillebælt. Dette er et problem, da bakterierne fungerer bedst med en konstant mængde af klorid i spildevandet. Variationer i klorid gør, at bakteriernes nedbrydningsproces af de forskellige stoffer i spildevandet er nedsat for en periode. Når indholdet af klorid er konstant igen tilpasser bakterierne sig og deres nedbrydningsproces går tilbage til normal kapacitet.
	\end{itemize}
	\item Der er ingen forfældning i rensningsanlægget pga. luftgener. 
	\begin{itemize}
		\item Har tidligere udtaget primær slam ved bundfældning, det blev stoppet pga. høj gas udvikling, da dette gav lugtgener.
		\item Der planlægges igen at udtage primær slam ved en filtreringsproces for at undgå lugtgener.
	\end{itemize}
	\item Fosfor kommer hovedsageligt fra industri ca. 300-500 kg/døgn under normal drift. Fosfor er nødvendigt da dette indgår i processen til at nedbryde kvælstof.
	\item Varierende indhold af klorid i spildevandet er et problem for bakterierne, dette er især et problem under 10 mg/l.
	\item Rensningsanlægget har en kapacitet på 420.000 People Equivalent (PE), PE svarer til 120 COD/døgn eller 0,2 mg/l/døgn for en person.
	\item Når der er tørvejr er der et typisk flow på 800-1200 $m^3/time$ ind til rensningsanlægget.
\end{itemize}
Det ideelle scenarie er, 
\begin{itemize}
		\item Konstant flow og koncentrat
		\item Fast indhold af koncentrat (Klorid bl.a.)
		\begin{itemize}
			\item Nødvendigvis ikke et lavt indhold
		\end{itemize}
\end{itemize}	
Prioriteringer i forhold til forstyrrelse i styring af ledningsnetværk. 
\begin{itemize}
	\item Små klorid variationer
	\item Slam/bakterier nok til at kunne omsætte kvælstof
	\begin{itemize}
		\item Dette reguleres der for i rensningsanlægget. 
	\end{itemize}
	\item Der skal være en hvis mængde kulstof
	\begin{itemize}
		\item Hvis spildevand flowet er konstant, er dette ikke et problem for rensningsanlægget.
	\end{itemize}
	\item Små flow variationer
	\item Lav opholdstid i bassiner
\end{itemize}

