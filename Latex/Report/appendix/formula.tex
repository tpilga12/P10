\newpage
\section{Formulas}\label{app:formulas}
In this section the equations for calculation the area, water width and flow in a pipe will be shown.
%\input{/report/appendix/tikz/circle}


In figure \ref{fig:calc_water_pipe_width} a illustration a cross view of a circular pipe is shown
\begin{figure}[H]
	\centering
	\includegraphics[width=0.20\textheight]{report/appendix/figures/calc_water_pipe_width.pdf}
	\caption{Cross section view of a circular pipe. Where d is the diameter, b is the width of the water in a given height, h is the height and the crossed section in the bottom illustrates the area of the water in the pipe.}
	\label{fig:calc_water_pipe_width}
\end{figure}

The area of the water in a circular pipe is calculated with the following \cite{ikke_stationear}: 
\begin{equation}%\label{eq:calc_area_open_channel}
	A = \frac {d^2}{4} \cdot acos \left(\frac{\frac{d}{2}-h}{\frac{d}{2}}\right)-\sqrt{h\cdot (d-h)}\cdot  \left(\frac{d}{2}-h\right)
\end{equation}

The water width in a pipe to a given height is calculated with the following equation \cite{ikke_stationear}:
\begin{equation}
	b = 2 \cdot \sqrt{-h+h\cdot d}
\end{equation}

The flow in a filled pipe can be calculated with the equation describing the friction in a pipe \cite{ikke_stationear}:
\begin{equation}
	S_f = \frac{n^2 Q^2}{A^2R^{4/3}}= \frac{n^2 v^2}{R^{4/3}}
\label{Manning_formulav2}
\end{equation}
Solved for Q:

\begin{equation}
	Q=\frac{1}{n} \cdot A\cdot R^{2/3} S_f^{1/2}			
\end{equation}

Where n is Mannings roughness factor $[sm^{-1/3}]$, which is 0,0139 for concrete pipe. R is the hydraulic radius [m], A is the area $[m^2]$ and Q is the flow $[m^3/s]$. Taking the square root on both sides and assuming that the equation is for a filled pipe, the following can be written:

\begin{equation}
  Q_{f} = \frac{1}{n} A_{f} R_{f}^{0.635}S_f^{0,5} 
\end{equation}


Where the subscript f denotes for filled pipe. For a filled pipe the area is the cross section for a circle is $\pi \cdot r^2$. The hydraulic radius is the wetted area divided by the wetted perimeter. As the wetted area is the area for a circle and as the wetted perimeter is equal to the perimeter of a circle, $2\pi r$, the following can be written: 
\begin{equation}
		Q_{f}= \frac{1}{n}\pi r^2\left(\frac{\pi r^2}{2\pi r}\right)^{0,635} S_f^{0,5} 
\end{equation}

This can be simplified to the equation for a filled pipe:

\begin{equation}%\label{eq:qf_for_flow}
	Q_{f} =\frac{1}{n}\pi\left(\frac{d}{2}\right)^2\left(\frac{d}{4}\right)^{0.635} S_f^{0,5}% -3.02 \cdot ln\left(\frac{0.74\cdot 10^{-6}}{d\sqrt{d\cdot I_e}}+\frac{k}{3.71\cdot d}\right)d^2\sqrt{d\cdot I_e}
\end{equation}


The flow in a pipe given a height can be calculated with the following equation \cite{ikke_stationear}:
\begin{equation}%\label{eq:calc_for_flowv2}
 	Q = \left(0.46-0.5 \cdot cos\left(\pi \frac{h}{d}\right)+0.04\cdot cos\left(2\pi\frac{h}{d}\right)\right)\cdot Q_f
\end{equation}

