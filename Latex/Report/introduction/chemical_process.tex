\section{Chemical process}\label{se:chemical_process}
This section will analyse the chemical process that wastewater undergoes from the water is used to it is cleaned at the water treatment plant. 


% \begin{figure}[H]
% \centering
% \includegraphics[width=1\textwidth]{report/introduction/sewer_pictures/sewer_overview_from_user_to_cleaned_water.png}
% \caption{Arbejdsblad billede.}
% \label{fig:sewer_overview_of_the_different_parts}
% \end{figure}
% \fxnote{Illustrering af fig 1,1 hvitved og dertil beskrivelse af den, for at give et overblik over kloakker og dens processer}

\subsection{Chemical reactions in a sewer}\label{subse:chemical_reactions_in_a_sewer}
A wastewater treatment plant does not only contain what is discharged into the sewer from the industry and households but also the chemical and microbial reactions that occurs in a sewer. These reaction occurs as redox reactions between the different compounds. Redox reaction is the transfer of electrons between two compounds at a atomic or molecular scale. By transferring electrons from one compound to another new compounds will rise, such as hydrogen sulfide which is know for it malodorous smell of rotten eggs. \fxnote{Hvad producere Heteotropic Microorganisms} Theses reactions are determined by the electron acceptors that are present in the wastewater. The electron acceptor is the compound that receives electrons in a redox reaction. Examples of dissolved acceptors are oxygen ($O_2$), nitrate ($NO^-_3$) and sulfate ($SO^{2-}_4$), which determines whether aerobic, anoxic or anaerobic processes may occur. The redox reaction reduces these three compounds in the wastewater by changing them to new compounds as water $H_2O$, molecular nitrogen ($N_2$) and hydrogen sulfide ($H_2S$). Redox reactions are determine to a great extend by the design of the sewer where different conditions such as aerobic, anaerobic and anoxic exist, where the last only occurs if nitrate is artificially added to the wastewater. If the process is aerobic the typical characteristics for the sewer are either partly filled gravity sewer or a aerated pressure sewer which means that there are free oxygen ($O^+$), and these will be connected with hydrogen to create water. In the case of anoxic the wastewater is pumped and there are added nitrate resulting in molecular nitrogen. In a anaerobic sewer the characteristic of the sewer is either a pressure sewer or a full flowing gravity sewer, then the redox reaction will result in hydrogen sulfide. Therefore sewers can actively be designed to achieve a specific process. This knowledge is used to make simulation that is able to express the chemical and microbial reaction which occur in the wastewater. As previous mention these are examples of some of the reactions that occur in the wastewater. Furthermore turbulence in the flow of the wastewater affects reaeration and can release odorous and corrosive particles into the atmosphere.    

% \begin{figure}[H]
% \centering
% \includegraphics[width=1\textwidth]{report/introduction/pictures/sewer_overview_of_the_different_parts.png}
% \caption{Illustrates how the wastewater flows from the industry and households to the treatment plant. \fxnote{Ny tegning}}
% \label{fig:sewer_overview_of_the_different_parts}
% \end{figure}

\subsection{Wastewater treatment plant}\label{subse:Wastewater treatment plant}
At the wastewater treatment plant the wastewater undergoes several process before the chemicals, dirt, etc. is removed from the water and it is lead back into the nature. The first stage in clearing the wastewater is screening, where larger objects are remove from the wastewater. Some of the objects that are filtered from the wastewater are bottles, plastic bags, diapers, etc. all items that are to big that would either block or damage the equipment. 

Stage two is the primary treatment where the separation of organic matter (human waste) from the wastewater. By leading the wastewater into a large settlement tank the organic matter will sink to the bottom of the tank. The matter that have sedimented is now called sludge. At the bottom large scrappers are scrapping the floor moving the sludge to the center of the tank where it is pumped away for further treatment. From here the separated water is pumped into the secondary treatment. 

In this stage the water is pumped into large rectangular tanks where air is pumped into the water to make the bacteria consume the sludge that have passed on from the previous stage. When the bacteria have consumed the sludge it will start to sediment at the bottom of the tank. This process takes 3-6 hours.  

After these treatment processes there are still some diseases in the water. If the water is lead in to sensitive or fragile ecosystems the water is further treated. It is disinfected for at least 20-25 minutes in tanks where a mixture of chemicals are used to cleanse the remaining water before leading it into the environment. If the water is not lead to a fragile ecosystem the last phase is passed through a settlement tank where the remaining particles will sediment at the bottom of the tank creating more sludge. From here the water is lead through a filter the remove any additional particles. Hereafter the water is lead into the environment. 

The sludge that is collected in the process undergoes further treatment, where the remaining water in the sludge is separated from it. This water is lead back to the wastewater treatment process. The sludge is sent back to the environment or used for agricultural use.





