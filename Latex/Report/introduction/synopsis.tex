This project deals with control of wastewater management in sewers, where Fredericia has been used as a case. In Fredericia, there are multiple large companies which dispose large amounts of wastewater into the sewer. This results in an input to the WWTP that variates which is not optimal for the WWTP. Therefore this project looks at a solution to obtain a sewage output where the variations are minimized, which results in the following problem statement: \textit{How can a simulation environment be constructed, which mimic the behavior of a real sewer system, where MPC is utilized as the control scheme to obtain stable sewage output such that optimal performance can be obtained from a WWTP.} Therefore models of a sewer networks were constructed to describe the flow in a sewer. These are used to build a simulation model, that is able to simulate a sewer network that includes, pipe, interconnections, and tanks. The simulation model is able to simulate a sewer network with flow and concentration. To control the input into the WWTP in an optimal manner an MPC controller is used. However, the MPC controller is not able to keep the variations of the flow minimized and therefore needs further investigation.            






% This project addresses the problem with hysteresis in control valves in mixing loops. These mixing loops are widely used in larger builds to supply them with hot or cold water for e.g. in-floor heating. Within these mixing loops, a control valve is placed to regulate the temperature of the water. These valves are regulated with an actuator which is subject to hysteresis. It could either be through wear and tear on the gears of the actuator or it could be due to cheap materials used designing it. Therefore a study began to understand what is causing hysteresis. Hereafter a state of the art is performed to investigate what has already been done to compensate for this problem. One solution was of particular interest as it was able to both detect and compensate for hysteresis. With a possible solution in mind different models was established to create a simulation to test this solution on. These models were able to mimic the behavior that was seen in the laboratory. The solution was tested in the simulation without luck, and therefore another method was developed. This method was tested in the simulation where it was deemed to be verified. Therefore it was tested on the testing platform where it showed an improvement. However further testing is needed to conclude on the robustness of this method.      


% This report addresses the work with pressure management in a potable water supply network.
% To supply water to the consumers, a large amount of energy is needed to maintain the correct pressure in the water distribution network. Unfortunately the high pressure leads to water leakages, typically cause by the wear and tear. To partially mitigate this problem the network is divided into smaller interconnect areas, with pressure control. Therefore it is examined how to regulate the pressure in these areas, which leads to the following problem statement:
% \begin{center}
% \label{ProblemStatement_synopsis}
% \textit{How can multiple pressure management areas be regulated when these are interconnected and still maintaining a certain water pressure at the critical points?}
% \end{center}
% A dynamic model of a physical system was derived and used to design a cascade controller, to regulate the pressure in two interconnected pressure management areas. 
% \\
% \\
% The cascade controller were implemented in Simulink RealTime workshop and tested on the system. The tests showed that the system were slower than calculated, but were still able to keep a reference value at the critical points.



% Several incidents of death by drowning happens every year in Denmark and while the JRCC has a rescue rate of 94,6 \% improvment is always necessary. Therefore it is examined how to increase the effectivnes of the SAR missions which leads to the following problem statement:
% \begin{center}
% \label{ProblemStatement}
% \textit{How can an electronic system be designed to reduce the search time during rescue missions at sea by increasing the width of the search sweeps?}
% \end{center}  
% An experimental solution was deducted for practical reasons whereby a quadcopter should be able to hold a relative position to a transmitter by using radio and ultrasound for distance measuring, PD controllers for movement and PPM modulation for communication.   
% \\
% \\
% Every module of the system were implemented except two of the PD controllers due to wrong input although they worked individually. This gave the conclusion that the system was able to hold a relative position in one of the three movement axes.

% % This projec examine the design and construction of a control system for a quadcopter that will maintain a distance to a search and rescue helicopter to reduce the search time by increasing the width of a sweep.\\
% % A requirement specification is presented based on an functionality and technology analysis, which leads to a system design. In addition to the control system this paper examine PPM modulation and ultrasound.\\
% % One of the three designed control system for the final product is hold against an accepttest, which confirms it will function under a static environment although the overshoot is greater than desired.