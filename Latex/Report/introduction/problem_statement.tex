\newpage
\section{Problem statement}
\label{sec:problem_statement}

Based on the information in the previous sections, which states that several problems occurs during wastewater treatment, the problems can be summed to the following points:

\begin{enumerate}
\item Flow variations due to large industries and natural phenomenons
\item Concentration variations due to large industries and natural phenomenons
\begin{enumerate}
	\item Chloride variations
	\item Phosphor variations
	\item Nitrogen variations
\end{enumerate}
\end{enumerate}

To be able to implement countermeasures towards these problems information is needed about the flow of wastewater in the sewer lines. As measurements is rarely available for entire sewer networks or even parts of it, an ideal solution would be to construct a simulation environment where different scenarios can easily be setup \ref{app:resume}. Furthermore a control scheme is needed to be able to implement an efficient countermeasure. It has been decide to utilize model predictive control (MPC) as the control scheme for this project. This control scheme excels in obtaining optimal control action, in environments where stochastic events occurs, making it ideal for this project.
From this a problem statement can be formulated: \\
\begin{center}
\textit{How can a simulation environment be constructed, which mimic the behavior of a real sewer system, where MPC is utilized as the control scheme to obtain stable sewage output such that optimal performance can be obtained from a WWTP.}
\end{center}