\chapter*{Preface}
This report has been created by Jacob Naundrup Pedersen. The project is performed on the 3rd semester of the master control and automation at Aalborg University. The project is constructed in an internship at Grundfos. Grundfos has contributed with the test setup for the project. The student has followed two courses at Aalborg University, non-linear systems and machine learning. 

The report is intended for people with a background knowledge corresponding to a third-semester master student at Control and Automation, Aalborg University. The following programming languages MATLAB and Simulink are used in the project. All graphical elements in the report are constructed by the author. Otherwise, a reference to the source, is stated in the figure text.

Sources are indicated by [name,year], and can be found in the bibliography list at the given [name,year].

% This report is drafted by group 16gr634, a sixth semester group consisting of three persons studying EIT at Aalborg university. The purpose of this report is to document the work of the design and development of a control system for a water distribution network. It is a Bachelor project in Control Engineering and started on February 1. with a submission deadline on May 25. The group has parallel with the project participated in the following courses: "Matrix computations and 
% Convex optimization", "Introduction to probability theory and Statistics" and "Control Engineering".

% This report investigates the possibility to regulate a system of interconnected pressure management area using classic controllers. An analysis of a physical system leads to a dynamic model. Based on a requirement specification, a system design for a regulator, to control the physical system is implemented, and is held against the requirement specification in the final accepttest.
% %
% %The results for pressure will be calculated in meter water column. The reason for this is, that the results would either be very large or very small, so to make it more suitable for computer calculations it is converted to mWc.

% The report is drafted in cooperation with the supervisor:
%  \begin{itemize}
%  \item Carsten Skovmose Kallesøe, professor at Aalborg university.
%  \item Tom Nørgaard Jensen, post doc at Aalborg university.
%  \end{itemize}  
 
% The figures in the report is produced by the group unless a source is specified. Sources are indicated by [author,year] and can be found in the bibliography. Appendix is indicated by A.number or [Appendix/filename] on the CD. The paper is structured in chapters, sections and subsection. Every figure, table, equation and code is separately numbered continuously. Figures can be diagrams, flowcharts and graphs. The following is placed on the CD:
%  \begin{itemize}
%  	\item Calculations
%  	\item Datasheet
%  	\item MATLAB files
% % 	\item Simulink files
%  	\item Sources
% % 	\item Video 
%  \end{itemize}
% This paper is drafted by group 15gr512, a fifth semester group consisting of five persons studying EIT at Aalborg university. The purpose of this paper is to document the work of the design and development of a control system for a quadcopter. The theme of the semester is "Digital and Analog Systems Interacting with the Surroundings" and started on September 2. with a submission deadline on December 17. The group has parallel with the project participated in the following courses: "Signal Processing", "Modeling and Control" and "Communication in Electronic Systems". 

% This paper investigates the possibility to improve the search time during rescue mission at open sea by using an electronic system. An analysis of such system functionality and technology become the basis for a requirement specification. A system design with the interfaces between the modules leads to the development of a prototype, which is held against the requirement specification in the final accepttest.
% %
% %The paper is drafted in cooperation with the supervisor:
% %\begin{itemize}
% %\item Kirsten Mølgaard Nielsen, lektor at Aalborg university.
% %\item Tom Søndergaar Pedersen, lektor at Aalborg university.
% %\end{itemize}

% The group would like to thank the following person for help regarding Vicon:
% \begin{itemize}
% \item Karl Damkjær Hansen, post doc at Aalborg university.
% \end{itemize}

% The group would like give a special thank to:
% \begin{itemize}
% \item Emil Mürer who spent time and rescources designing and creating plastic conical cones.
% \end{itemize}

% The figures in the paper is produced by the group unless a source is specified. Sources are indicated by [author,year] and can be found in the bibliography. Appendix is indicated by A.number or [Appendix/filename] on the CD. The paper is structured in chapters and sections, every figure, table, equation and code is seperately numbered continuously. Figures can be diagrams, flowcharts and graphs. The following is placed on the CD: 
% \begin{itemize}
% 	\item Altium files
% 	\item Code
% 	\item Datasheet
% 	\item LTspice files
% 	\item Matlab files
% 	\item Video 
% 	\item Component list
% \end{itemize}


\vfill

\begin{table}[H]
	\centering
		\begin{tabular}{c c }
			\underline{\phantom{mmmmmmmmmmmmmmmmmmm}}       \\
			Jacob Naundrup Pedersen \\
		\end{tabular}
		
\end{table}

%\begin{tabular}{ |l l l| }
%\hline
%Symbol & Description & Unit \\ \hline
%\multirow{1}{*}{} &  &  \\ 
%\multirow{1}{*}{} &  &  \\ 
%\multirow{1}{*}{} &  &  \\ 
%\multirow{1}{*}{} &  &  \\ 
%\multirow{1}{*}{} &  &  \\ 
%\multirow{1}{*}{} &  &  \\ 
%\multirow{1}{*}{} &  &  \\ 
%\multirow{1}{*}{} &  &  \\ 
%\multirow{1}{*}{} &  &  \\ 
%\multirow{1}{*}{} &  &  \\ \hline
%\end{tabular}