\chapter{Introduction}
\label{ch:introduction}

Sewers were created to solve the seemingly simple problem of removal of wastewater. The first sewers, registered, dates back to 7000 B.C. in urban settlements and were created to remove wastewater from houses and surface runoff created by precipitation. To avoid clogging and wear of the sewers grit chambers was constructed. They work by slowing the flow of sewage in long narrow channels making the solids, such as sand, end up as sediments in the channels due to gravity. Complexity of sewers increased in ancient Rome where large underground systems were created leading to the main sewer system called "Cloaca Maxima" making it possible to have latrines with running water within households, though mostly made available for the rich \cite{Sewer_processes}.

Waste were still thrown onto streets as the population, without immediate access to a latrine in their household, during night time did not want to put in the effort to properly dispose of the waste. Because of this the ancient Rome suffered from illnesses related to waste lying in the streets. The hygienic aspect of proper disposal of wastewater in relation to drinking water were not considered until the 19th century, where several European cities saw large outbreak of cholera causing the deaths of millions \cite{Sewer_processes}. 

The growth in waste furthermore caused the expansion of 26 km sewer network in Paris to 600 km during the 19th century. But it is not until the start of the 20th century that the chemical and microbial processes in sewers are considered. The microbial cause of cholera were identified by the German doctor Robert Koch in 1883, a discovery for which he in 1905 received the Nobel Prize in physiology and medicine. The growing industries and technological progress in the 20th century meant that more chemicals were disposed into the sewers having severe consequences for the organic life downstream of the receiving waters. Wastewater treatment plants were introduced to reduce the pollution, but several countries did not have any wastewater treatment plants before after World War II. Today disposal of sewage and setup of wastewater treatment plants is a given part of construction of new settlements, even in poor regions of the world \cite{Sewer_processes}.


% %% Skriv noget fra 1.1
% A sewer system is used for removing wastewater originating from households, industries and runoff from different urban areas. It is collected in the sewers and transported through the sewer network to a wastewater treatment plant, where the wastewater will go through a filtering process and thereafter be discharged into to a receiving water system. The European definition of a sewer network is: \textit{A sewer system is a network of pipelines and ancillary works that convey wastewater from its sources such as a building, roof drainage system, or paved area to the point where it is discharged into a wastewater treatment plant or directly into the adjacent environment (BS EN 752.1, 1996)}\fxnote{find kilden, dette er en citering fra biblen, hvor han refer til BS EN 752.1}. Sewer network consist of pipes (sewer lines) and different installations and structures such as inlets, manholes, drops, shafts and pumps.

% %% Skrevet udfra section 1.2.1 i biblen
% Sewer network date back to the beginning of urban settlements. These networks was used to remove either wastewater from houses or surface runoff in populated areas \fxnote{kilde}. Around 2500-2000 BC the settlement of the Indus Valley Civilization, located in the west Pakistan, buildings shows bathing and latrine facilities. Where a sewer system equipped with a grit chamber is used to remove unwanted materials from the water. A grit chamber is a long narrow tank designed to slow the flow and thereby solids such as sand will settle out of the water due to gravity. At the time these grit chambers was important for a proper function of the sewerage.

% %% Skrevet fra 1.2.3 og 1.2.4
% Up until the seventeenth century most large cities in Europe had underdeveloped drainage infrastructure. Hereafter the development of underground sewer networks started. Paris was one of the first to develop a efficient sewer network. In the period of the French revolution, around 200 years ago, the total length of sewer network in Paris was only 26 km long. This was extended to 600 km in 1887. Which indicate the need of removing wastewater from the growing cities. The reasons for constructing these underground sewers, was to collect the wastewater, due to the malodorous smell from open sewers, cesspools, privies and furthermore, freeing space in the densely packed streets of the populated cities, thereby giving space to roads, housing etc.
 
%% Skriv noget fra 1.2.5
