\section{Linearization}\label{se:linearization}
To be able to obtain a MPC control algorithm the system needs to be transform into state space form. In this section it will be elaborated how this is obtained.


The saint venant equation from section ??? is used. 
\begin{equation}\label{eq:linearization_Continuity}
\frac{\partial A(x,t)}{\partial t} + \frac{\partial Q(x,t)}{\partial x}=0
\end{equation}

The equation is expanded to the following:

\begin{equation}
	\frac{\partial A(x,t)}{\partial h}\frac{\partial h(x,t)}{\partial t} + \frac{\partial Q(x,t)}{\partial h}\frac{\partial h(x,t)}{\partial x}=0
\end{equation}

By applying the Preissmann scheme from section \ref{subse:preissmann_scheme} equation \ref{eq:preissmann_time_derivatie} and \ref{eq:preissmann_space_derivatie} on the partial derivate of h in terms of time and position, the following is obtained: 

\begin{multline}
	\frac{\partial A(x,t)}{\partial h} \left(\frac{1}{2}\frac{h_{j+1}^{i+1}-h_{j+1}^i}{\Delta t} +  \frac{1}{2} \frac{h_{j}^{i+1} - h_j^i}{\Delta t}\right) + \\ \frac{\partial Q(x,t)}{\partial h}\left(\theta \frac{h_{j+1}^{i+1}-h_j^{i+1}}{\Delta x}+(1-\theta)\frac{h_{j+1}^i - h_j^i}{\Delta x}\right)=0
\end{multline}
This equation can be written on a matrix form:

\begin{equation}
\begin{aligned}
	\begin{bmatrix}
		\underbrace{\frac{1}{2\Delta t}\frac{\partial A}{\partial h}-\frac{\theta}{\Delta x}\frac{\partial Q}{\partial h}}_{a}  & \underbrace{\frac{1}{2\Delta t}\frac{\partial A}{\partial h}+\frac{\theta}{\Delta x}\frac{\partial Q}{\partial h}}_{b} 
	\end{bmatrix}
	\begin{bmatrix}
		h_{j}^{i+1} \\
		h_{j+1}^{i+1}
	\end{bmatrix}
	= \\ -
	\begin{bmatrix}
		\underbrace{\frac{-1}{2\Delta t}\frac{\partial A}{\partial h}-\frac{(1-\theta)}{\Delta x}\frac{\partial Q}{\partial h}}_{c} & \underbrace{\frac{-1}{2\Delta t}\frac{\partial A}{\partial h}+\frac{\theta}{\Delta x}\frac{\partial Q}{\partial h}}_{d} 
	\end{bmatrix}
	\begin{bmatrix}
		h_{j}^{i} \\
		h_{j+1}^{i}
	\end{bmatrix}
	\end{aligned}
\end{equation}

We want the equation on the normal state space form.

\begin{equation}
	\dot{x} = Ax + Bu
\end{equation}
Where A is the system matrix, x is the state of the system, B is the input matrix and u is the input. 

\begin{equation}
\begin{aligned}
	   \underbrace{\begin{bmatrix}
	    	a & b \\
	    	  & a & b \\
	    	  &   & a & b  \\
	    	  &   &   & \ddots  & \ddots \\
	    	  &   &   &   &  &  \\
	    	  &   &   &   &  &  &  \\
	    	  &   &   &   &  &  &  &  \\
	    	  &   &   &   &  &  &  & a  &b\\
	    	  &   &   &   &  &  &  &  &  a\\
	   \end{bmatrix}}_{\xi}
	    \underbrace{\begin{bmatrix}
		h_{1}^{i+1} \\
		h_{2}^{i+1}\\
					\\
					\\
		\vdots		\\
					\\
					\\
					\\
		h_{m}^{i+1}\\
	\end{bmatrix}}_{\dot{x}}
	= -
	\underbrace{\begin{bmatrix}
	    	d &  \\
	    	c & d & \\
	    	  & c & d 		&   \\
	    	  &   &	\ddots   & \ddots  &  \\
	    	  &   &   &  	&  &  \\
	    	  &   &   &   &  &  &  \\
	    	  &   &   &   &  &  &  &  \\
	    	  &   &   &   &  &  &  &   &\\
	    	  &   &   &   &  &  &  & c &  d\\
	    \end{bmatrix}}_{A}
	    	\underbrace{\begin{bmatrix}
		h_{1}^{i} \\
		h_{2}^{i}\\
					\\
					\\
		\vdots		\\
					\\
					\\
					\\
		h_{m}^{i}\\
		\end{bmatrix}}_{x}
		\\
	+ \underbrace{\begin{bmatrix}
		C \\
		\\
					\\
					\\
		\vdots		\\
					\\
					\\
					\\
		0\\
		\end{bmatrix}}_{B}
		\underbrace{\begin{bmatrix}
		Q_{in} \\
		\\
					\\
					\\
		\vdots		\\
					\\
					\\
					\\
		0\\
		\end{bmatrix}}_{u}
	\end{aligned}
\end{equation}
Where m denotes the numbers of channels in the model.  
To obtain a state space for the matrix to the left needs to inversed, thereby obtaining the following equation:

\begin{equation}
	\dot{x} = -\xi^{-1} (Ax+Bu)
\end{equation}


