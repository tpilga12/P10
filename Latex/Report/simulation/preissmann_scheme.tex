\section{Preissmann scheme}\label{se:preissmann_scheme}
In this section, a numerical method for solving the Saint-Venant equations are chosen and elaborated on.
According to table \ref{momentum_approximations} various approximations of the momentum equation can be used. Common for the most part of them is that boundary conditions is needed up- and downstream. The problem is that downstream boundary conditions is not always known or require extensive information of the WWTP. It is therefore decided to utilize the kinematic wave approximation, as the hydraulic conditions of the WWTP is considered out of scope for this project.  
%Argument for at bruge numeriske metode til at løse saint venant eq.

The numerical method chosen, for solving the Saint-Venant equations, is the Preissmann scheme which is based on the box scheme. Other methods exist such as Lax scheme, Abbot-Ionescu scheme, leap-frog scheme, Vasiliev scheme, however, the Preissmann scheme is known for its robustness. Basically, by using the Preissmann scheme the Saint-Venant equations can be discretized, and thereby utilized to simulate the flow and height throughout a pipe \cite{cunge1980practical}.

In section \ref{se:hydraulics_of_sewer_line} the Saint Venant equations for conservation of mass and momentum are derived, they are also shown below.

\begin{equation}\label{eq:saintbernard_mass_preiss}
\frac{\partial A(x,t)}{\partial t} + \frac{\partial Q(x,t)}{\partial x}=0
\end{equation}

\begin{equation}\label{eq:saintbernard_momentum_preiss}
\frac{1}{gA} \frac{\partial Q}{\partial t} +\frac{1}{gA}\frac{\partial}{\partial x} \left( \frac{Q^2}{A} \right) + \frac{\partial h}{\partial x} + S_f - S_b = 0
\end{equation}

%Skriv noget om boundary og initial conditions

In figure \ref{fig:preissmann_grid_scheme} a single mesh for the Preissmann scheme is illustrated.

\begin{figure}[H]
\centering
%\includepdf[pages=1]{report/simulation/pictures/preissmann_scheme.pdf}
\includegraphics[width=.7\textwidth]{report/simulation/pictures/preissmann_scheme.pdf}
\caption{Preissmann non-staggered grid scheme.}
\label{fig:preissmann_grid_scheme}
\end{figure}

Where $\theta$ is a weighting parameter ranging between zero and one, j is an index of cross section and i is an index of time. The mesh contains four nodes, (j,i), (j+1,i), (j,i+1) and (j+1,i+1), however in the implementation the dimension of the grid is $\Delta t \times \Delta x$ for $0 \leq x \leq L$ and $0\leq t$. Where L defines the length of the pipe section. The derivatives in equations \ref{eq:saintbernard_mass_preiss} and \ref{eq:saintbernard_momentum_preiss} are calculated as an approximation at the point P, which is in the middle of the interval of $\Delta x$.% and is also the reason why the Preissmann scheme is corresponding to a box scheme.
 The difference between the box scheme and the Preissmann scheme is that the point P should always be located at $\Delta x/2$, meaning that the point can only move along the time axis within this mesh by adjusting the weighting parameter $\theta$. The effect of this weighting parameter will be elaborated in subsection \ref{subse:stability_and_precision}.
 An arbitrary function $f_p(x,t)$ calculated at point P is approximated by \cite{numerical_modeling}.

\begin{equation}\label{eq:approximated_function}
    f_P \approx \frac{1}{2} (\theta \cdot f_j^{i+1}+(1-\theta)f_j^i)+\frac{1}{2}(\theta\cdot f_{j+1}^{i+1}+(1-\theta)f_{j+1}^i)
\end{equation}
The numerical approximation for the derivatives in equations \ref{eq:saintbernard_mass_preiss} and \ref{eq:saintbernard_momentum_preiss} for time and length are shown below \cite{numerical_modeling}.

\begin{equation}\label{eq:preissmann_time_derivatie}
    \frac{\partial f}{\partial t}\bigg \rvert_P \approx \frac{1}{2}\left(\frac{f_j^{i+1}-f_j^i}{\Delta t}+\frac{f_{j+1}^{i+1}-f_{j+1}^i}{\Delta t}\right)
\end{equation}

\begin{equation}\label{eq:preissmann_space_derivatie}
    \frac{\partial f}{\partial x}\bigg \rvert_P \approx (1-\theta)\frac{f_{j+1}^i-f_{j}^i}{\Delta x}+\theta \frac{f_{j+1}^{i+1}-f_{j}^{i+1}}{\Delta x}
\end{equation}

The approximations of equation \ref{eq:preissmann_time_derivatie} and \ref{eq:preissmann_space_derivatie} can be utilized on the derivative terms in the Saint-Venant equations to achieve the following for equation \ref{eq:saintbernard_mass_preiss} and \ref{eq:saintbernard_momentum_preiss} respectively:

\begin{equation}\label{eq:continuity_eq_preissmann}
    \theta \frac{Q_{j+1}^{i+1}-Q_j^{i+1}}{\Delta x}+(1-\theta)\frac{Q_{j+1}^i - Q_j^i}{\Delta x}+
    \frac{1}{2}\frac{A_{j+1}^{i+1}-A_{j+1}^i}{\Delta t} + \frac{1}{2} \frac{A_{j}^{i+1} - A_j^i}{\Delta t} = 0
\end{equation}

% \begin{multline}
%     \frac{1}{2} \left(\frac{Q_{j+1}^{i+1}-Q_{j+1}^i}{\Delta t}+\frac{Q_{j}^{i+1} - Q_j^i}{\Delta t}\right) + \frac{\theta}{\Delta x} \left(\left(\frac{Q^2}{A}\right)_{j+1}^{i+1}-\left(\frac{Q^2}{A}\right)_{j}^{i+1}\right) + \\ \frac{1-\theta}{\Delta x}\left(\left(\frac{Q^2}{A}\right)_{j+1}^{i}-\left(\frac{Q^2}{A}\right)_{j}^{i}\right)+gA_p\theta \left(\frac{h_{j+1}^{i+1}-h_j^{i+1}}{\Delta x}\right)+ \\ gA_p(1-\theta)\left(\frac{h_{j+1}^{i} - h_j^i}{\Delta x}\right)+\left(\frac{g\cdot n_M^2}{R^\frac{4}{3}}\frac{|Q|Q}{A}\right)_P = 0
% \end{multline}
\begin{equation}\label{eq:Momentum_eq_preissmann_discrete}
\begin{aligned}
    &\frac{1}{gA_p}\left(\frac{1}{2} \left(\frac{Q_{j+1}^{i+1}-Q_{j+1}^i}{\Delta t}+\frac{Q_{j}^{i+1} - Q_j^i}{\Delta t}\right)\right) +  \\ \vspace{2mm}
   &\frac{1}{gA_p}\left(\frac{\theta}{\Delta x} \left(\left(\frac{Q^2}{A}\right)_{j+1}^{i+1}-\left(\frac{Q^2}{A}\right)_{j}^{i+1}\right) +  \frac{1-\theta}{\Delta x}\left(\left(\frac{Q^2}{A}\right)_{j+1}^{i}-\left(\frac{Q^2}{A}\right)_{j}^{i}\right) \right) + \\ \vspace{2mm}
   &\theta \left(\frac{h_{j+1}^{i+1}-h_j^{i+1}}{\Delta x}\right)+ (1-\theta)\left(\frac{h_{j+1}^{i} - h_j^i}{\Delta x}\right) + \\ \vspace{2mm}
   &S_f-S_b= 0
    \end{aligned}
\end{equation}

By discretizing the Saint-Venant equations they can be used in a simulation to calculate parameters for the pipe model. The mesh shown in figure \ref{fig:preissmann_grid_scheme} is used to calculate the value of the node (j+1,i+1) by knowing the previous values in time and length (j,i), (j+1,i) and (j,i+1). Therefore some initial conditions must be known to be able to calculate the parameters for the pipe at the first iteration. The boundary conditions for the flow, at t=0, must be known throughout the pipe. Furthermore, the inflow to the pipe for each iteration must be known. This is illustrated with circles in figure \ref{fig:preissmann_grid_scheme_exampel}.

\begin{figure}[H]
\centering
\includegraphics[width=.75\textwidth]{report/simulation/pictures/preissmann_scheme_iteration}
\caption{Preissmann non-staggered grid scheme example of iteration pattern.}
\label{fig:preissmann_grid_scheme_exampel}
\end{figure}
%% Dette skal omformuleres
By knowing inflow and specifications of the pipe, height can be calculated at the initialization nodes. With equation \ref{eq:continuity_eq_preissmann}, the flow at (j+1,i+1) can be calculated by knowing the flow and height at the previous nodes (j,i), (j+1,i) and (j,i+1) as illustrated with the box in the left bottom corner in figure \ref{fig:preissmann_grid_scheme_exampel}.

%For the Preissmann scheme to be numerical stable the $\theta$ parameter is $\theta \leq 0,5$. Does closer $\theta$ is to 0,5 the more accurate it is, however it is also more likely to be unstable, therefore from \cite{theta_decision} it is suggested to place $\theta$ in the range of $0,55 \leq \theta \leq 0.65$ for practical analysis.

%In the following section, the calculating of the flow and height in each node will be explained.

%\subsection{Iteration scheme} \fxnote{This subsec needs to get canned, hvorfor skal dette afsnit på dåse? xD} In this section, it will be elaborated how the calculation of the flow and height in each iteration is done \cite{ikke_stationear}.

The discretized continuity equation \ref{eq:continuity_eq_preissmann}, stated below, is solved for the desired flow in equation \ref{eq:continuity_solve_flow},
\begin{equation}
    \theta \frac{Q_{j+1}^{i+1}-Q_j^{i+1}}{\Delta x}+(1-\theta)\frac{Q_{j+1}^i - Q_j^i}{\Delta x}+
    \frac{1}{2}\frac{A_{j+1}^{i+1}-A_{j+1}^i}{\Delta t} + \frac{1}{2} \frac{A_{j}^{i+1} - A_j^i}{\Delta t} = 0
\end{equation}

\begin{equation}\label{eq:continuity_solve_flow}
    Q_{j+1}^{i+1} = - \frac{1}{2\theta}\cdot\left(A_{j+1}^{i+1}-H\right)\cdot\frac{\Delta x}{\Delta t}
\end{equation}
Where H is a parameter for the previous flows and areas in time and section, which are known, as these has been calculated, as shown in figure \ref{fig:preissmann_grid_scheme_exampel}. 
\begin{equation}
    H = \left(2\cdot(1-\theta)\cdot Q_j^i-2\cdot(1-\theta)\cdot Q_{j+1}^i+2\theta Q_j^{i+1}\right)\cdot\frac{\Delta t}{\Delta x}- A_{j}^{i+1}+A_j^i+A_{j+1}^i
\end{equation}

It has been chosen to set the slope of the water surface identical to the pipe bed, $S_f = S_b$. By doing this the calculation of the flow throughout the pipe has been simplified, thus some limitation about the dynamics of the water flow are excluded, which is explained in section \ref{se:hydraulics_of_sewer_line} and also shown in table \ref{momentum_approximations}. Therefore, instead of using the second discretized Saint-Venant equation \ref{eq:Momentum_eq_preissmann_discrete} an equation for calculating the flow in a circular pipe will be used to find the flow and area used in the Preissmann scheme and can be seen in appendix \ref{app:formulas} \cite{ikke_stationear}:\fxnote{Skal der står mere om valg af at sætte Ie = Ib}

\begin{equation}\label{eq:calc_for_flowv2}
     Q = \left(0.46-0.5 \cdot cos\left(\pi \frac{h}{d}\right)+0.04\cdot cos\left(2\pi\frac{h}{d}\right)\right)\cdot Q_f
\end{equation}

This equation describes the flow in a circular pipe by knowing the diameter, d height, h and the flow for a filled pipe $Q_f$. %Where $Q_f$ is calculated as \cite{ikke_stationear}:

\begin{equation}\label{eq:qf_for_flow}
    Q_f =72\cdot \left(\frac{d}{4}\right)^{0.635}\pi\cdot\left(\frac{d}{2}\right)^2\cdot S_f^{0,5}% -3.02 \cdot ln\left(\frac{0.74\cdot 10^{-6}}{d\sqrt{d\cdot I_e}}+\frac{k}{3.71\cdot d}\right)d^2\sqrt{d\cdot I_e}
\end{equation}
$Q_f$ is calculated from Mannings equation shown in \ref{Manning_formula} and can be seen in equation \ref{Manning_formulav2} in appendix \ref{app:formulas}.

In equation \ref{eq:continuity_solve_flow} the flow $Q_{j+1}^{i+1}$ is a function of the unknown area $A_{j+1}^{i+1}$, and by subtracting the flow on each side the following is achieved,

\begin{equation}\label{eq:continuity_zero_eq}
        0=-Q_{j+1}^{i+1}  - \frac{1}{2\theta}\cdot\left(A_{j+1}^{i+1}-H\right)\cdot \frac{\Delta x}{\Delta t}
\end{equation}
By naming the right hand side of equation \ref{eq:continuity_zero_eq} for V the following is obtained,

\begin{equation}\label{eq:continuity_V}
        V=-Q_{j+1}^{i+1}  - \frac{1}{2\theta}\cdot\left(A_{j+1}^{i+1}-H\right)\cdot \frac{\Delta x}{\Delta t}
\end{equation}

%Equation \ref{eq:continuity_V} can be solved by finding the zero for the function V using newton method. 
There are still two unknowns in equation \ref{eq:continuity_V}, $Q_{j+1}^{i+1}$ and $A_{j+1}^{i+1}$. $Q_{j+1}^{i+1}$ can be replaced with equation \ref{eq:calc_for_flowv2} for calculating the flow. Equation \ref{eq:calc_for_flowv2} is inserted into equation \ref{eq:continuity_V} and thereby the following is obtained,

\begin{equation}\label{eq:V_with_flow}
    V = -Q_f\cdot\left(0,46-0,5\cdot cos\left(\pi \frac{h_{j+1}^{i+1}}{d}\right)+0,04\cdot cos\left(2\pi\frac{h_{j+1}^{i+1}}{d}\right)\right)\frac{\Delta t}{\Delta x}-\frac{1}{2\theta}\left(A_{j+1}^{i+1}-H\right)
\end{equation}

Furthermore $Q_f$ from equation \ref{eq:qf_for_flow} is inserted into equation \ref{eq:V_with_flow},

\begin{multline}\label{eq:new_continuity_equation}
    V = -72\left(\frac{d}{4}\right)^{0.635}\pi\cdot\left(\frac{d}{2}\right)^2S_f^{0,5}\cdot \left(0,46-0,5\cdot cos\left(\pi \frac{h_{j+1}^{i+1}}{d}\right)+ 0,04\cdot cos\left(2\pi\frac{h_{j+1}^{i+1}}{d}\right)\right)\\ \frac{\Delta t}{\Delta x}-\frac{1}{2\theta}\left(A_{j+1}^{i+1}-H\right)
\end{multline}

V is now a function of the height $h_{j+1}^{i+1}$ as the height is the only unknown parameter for finding the area $A_{j+1}^{i+1}$ and the flow for the pipe, where the wetted area for a circular pipe is calculated with the following \cite{ikke_stationear}:

\begin{equation}\label{eq:calc_area_open_channel}
    A = \frac {d^2}{4} \cdot acos \left(\frac{\frac{d}{2}-h}{\frac{d}{2}}\right)-\sqrt{h\cdot (d-h)}\cdot  \left(\frac{d}{2}-h\right)
\end{equation}

%Equation \ref{eq:new_continuity_equation} can be solved by finding the zeros, where V is a function of the height, h.

To find a height in equation \ref{eq:new_continuity_equation} Newton's method is used. Newton's method is a method used to find better approximations to the roots of a real-valued function. The method requires a real-valued function f, the derivate f$'$, and a initial guess $x_0$, of the root of the function. If the assumption is satisfied and the initial guess is close then a better approximation can be obtained by:

\begin{equation}\label{eq:newtons_method_standard}
     x_1 = x_0 - \frac{f(x_0)}{f'(x_0)}
\end{equation} 

Where $x_1$ is the better approximation of the root, for the function f. Newtons method can be iterated until a satisfied root is obtained:


\begin{equation}\label{eq:newtons_method_standard}
     x_{n+1} = x_n - \frac{f(x_n)}{f'(x_n)}
\end{equation} 

By using the Newton method the roots of equation \ref{eq:new_continuity_equation} can be found, which will be a approximation of the height in the pipe. The approximation for the height is:

\begin{equation}\label{eq:newtons_method_height}
     (h_{j+1}^{i+1})_{k+1} =(h_{j+1}^{i+1})_{k} - \frac{V_k}{V'_k}
\end{equation}

Where k is the number of iterations, V$'$ is the differentiated of V with respect to height, $(h_{j+1}^{i+1})_{k}$ is an initial guess of the root and $(h_{j+1}^{i+1})_{k+1}$ is a better approximation of the height. This calculation is iterated until a satisfied approximation is achieved which fulfills the requirement,

\begin{equation}\label{eq:converge_newtons}
    \left(h_{j+1}^{i+1}\right)_{k}-(h_{j+1}^{i+1})_{k-1} < (\epsilon \cdot h_{j+1}^{i+1})_{k}
\end{equation}

Where $\epsilon$ is a small tolerance number, e.g. five-centimeter variation in water height. Equation \ref{eq:converge_newtons} calculates if the difference between the previous and the current height, if it is smaller than the tolerance, the calculation stops and returns the height. If not, the iteration scheme will stop and return an error. Thereby the water height can be found and the area of the water can be calculated with equation \ref{eq:calc_area_open_channel} \fxnote{afsnit om de forskellige ligninger til beregning af Areal, Ie,Ib,Qf,Q} and thereafter equation \ref{eq:continuity_solve_flow} can be used to calculate the flow of the node.

This calculation is performed for each node in the Preissmann scheme, therefore it is an iterative method of solving the flow for a pipe and in figure \ref{fig:flow_chart_iteration} a flow chart of how these equations are iterated is shown.
\begin{figure}[H]
    \centering
    \includegraphics[width=0.95\textwidth]{report/simulation/pictures/flow_chart_iteration.pdf}
    \caption{Flow chart of the iteration process to calculate the flow in each point.}
    \label{fig:flow_chart_iteration}
\end{figure}

The iteration scheme starts with setting the boundary conditions for the flow and height as seen in figure \ref{fig:preissmann_grid_scheme_exampel}, thereafter i and j are countered one up before starting the iteration process. Hereafter the calculation for H, h, A, and Q are conducted iteratively throughout the pipe. Finally, a check is made to investigate if the iteration scheme has been through the whole pipe before it steps one up in time. This iteration process will go on until the i = n, which means that all the inputs to the pipe have been calculated throughout the pipe.   

In the following stability and precision for the Preissmann scheme will be elaborated. %Where, how to chose $\Delta t, \Delta x$ and $\theta$ will be explained. 

\subsection{Stability} \label{stability}

One of the advantages of the Preissmann scheme is that if $\Theta \geq 0$ then stability is guaranteed. 

