\section{Structure}\label{sec:Structure}
For the simulation environment to be useful some basic functionality is needed. The simulation should be easy to setup and adjust if needed. 
Furthermore an easy way to view the result of the simulation is needed such that necessary adjustments can easily be made based on the result.
Based on this the design phase can be split in to three parts:

\begin{enumerate}
	\item Initialization
	\item Simulation
	\item Result
\end{enumerate}

The following will go into details of the design off, and the considerations made during the design phase, of the three above points.
As mentioned in section \fxnote{henvis til afsnit hvor tankene er foreslået som løsning, hvis der ikke er lavet den afgrænsning endnu} a solution to minimize flow and concentrate variations could be insertion of one or more tanks in the sewer network. 

The first thing to consider is the composition of components desired to simulate. Making the simulation able to simulate different setups than the one shown in figure \ref{fig:kloakgrid_simplified} a universal setup where different sizes of pipes with different slopes with selective side input is needed. 
The second thing to consider is that the environment should be brought to a steady state before the simulation starts. The reason for this is that if a linearized approach to the MPC scheme is chosen a linear model can be obtained. 

\subsection*{Initialization} 