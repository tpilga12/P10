
\chapter{Simulation solutions \& limitations }\label{ch:simulation_solution_and_limitation}
In this chapter different solutions will be discussed to find the most optimal solution to implement in Fredericia to limit the flow and concentration variations into the WWTP.

To limit the variations into the WWTP a mechanism is needed to hold back the wastewater. This could either be done by placing a valve within a pipe that would be able to limit the flow through the pipe.% This is done by opening or closing the valve and thereby the flow into the WWTP could be controlled. 
Another possibility is to store the wastewater in tanks and thereby control the output of the tank with a pump. 

Using a valve to hold back wastewater corresponds to using the pipe as a tank. This could lead to overflow if not controlled properly and thereby cause wastewater to flow onto streets and into the surrounding environment. %However, it would be easier to install valves throughout the sewer network in respect to dig a large hole for a tank in the city. 
This issue were discussed with the wastewater department in Fredericia and they pointed out that storage is limited due to the size of the sewer pipes and constant flow. 
%However, after the meeting with the people from Fredericia Spildevand og Energi A/S it was stated that it would not be possible to use the sewer as a storing device as the sewer would be filled up to rapidly.     

In this project, it is therefore decided to use one or more tanks as a solution to limit flow and concentration variations into the WWTP. %Using a tank will remove these fluctuations, as seen in figure \ref{fig:flow_input_wwtp} for flow, by controlling the output of the tank properly, thus having an input to the WWTP that is less disrupted by variations. %The output of the tank can be controlled to achieve a desired input to the WWTP, 
However, the tank must be controlled in a way where overflow in the tank is not permitted. %The output of the tank can either be controlled by a pump. %or a valve, depended on the geographical location on the tank. 
In addition, from the meeting at Fredericia it was informed, that if tanks is used as a solution, it is necessary to keep retention time of the stored wastewater in mind. The reason for this is, that if the wastewater is kept in the tank for a longer period of time, it will start to produce malodorous gas. This is due to oxygen depletion, as the environment in the tank will go from an aerobic to an anaerobic state. 

In Denmark sewers is made of concrete and the inner channel is constructed with a circular cross section area \cite{betonhaandbogen}. For this reason the simulation will be limited to work with sewer pipes of this form.


