
\chapter{Simulation solutions \& limitations }\label{ch:simulation_solution_and_limitation}
In this chapter different solutions will be discussed to find the most optimal solution to implement in Fredericia to limit the flow and concentration variations into the WWTP.

To limit the variations into the WWTP a mechanism is needed to hold back the wastewater. This could either be done by placing valves within the channels that would be able to limit the flow through a channel. This is done by opening or closing the valve and thereby the flow into the WWTP could be controlled. Another possibility is to store the wastewater in tanks and thereby control the output of the tank with e.g. a pump. 

Using a valve to control/hold back the flow is correspond to using the pipe as a tank. In extreme cases, this could lead to overflow if not controlled properly and thereby cause wastewater to overflow into small rivers and other safety measures. However, it would be easier to install valves throughout the sewer network in respect to dig a large hole for a tank in the city. But after the meeting with the people from Fredericia Spildevand og Energi A/S it was stated that it would not be possible to use the sewer as a storing device as the sewer would be filled up to rapidly.     

In this project, it is therefore decided to use a tank to limit the flow and concentration variation into the WWTP. Using a tank will remove these fluctuations, as seen in figure \ref{fig:flow_input_wwtp} for flow, by controlling the output of the tank properly, thus having an input to the WWTP that is less disrupted by variations. The output of the tank can be controlled to achieve a desired input to the WWTP, however, the tank must be controlled in a way where overflow in the tank is not permitted. The output of the tank can either be controlled by a pump or a valve, depended on the geographical location on the tank. In addition, from the meeting at Fredericia it was told, that in the case, where a tank was used as buffer in the system, it was needed to include retention time of the wastewater in the tank. The reason for this is, that if the wastewater is kept in the tank for a longer period, it will start to produces malodorous smells, due to the oxygen depletion, as the environment in the tank will go from an aerobic to an anaerobic environment. 

