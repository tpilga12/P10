\section{Introduction to control}
\label{control_intro}

As the title of the section suggests, this chapter is about to present the different attempts that were made during the design process to control the plant. All the approaches that were carried out are presented, even though some of the control techniques were later found not to be realistic, therefore neglected and not included in the paper. The necessity of introducing these attempts is to provide argument for the final solution and thereby come to a result that fits expectations the best. 

The introduction of the control starts with a general system and model description and continues onto a state-space representation. After describing the system, a concept is introduced for the disturbance which was implemented in simulation. This concept however, called exo-systems, was neglected because of the assumption that certain information are available from the disturbance. During the measurements the disturbance cannot be, or at least, would be very unrealistic to describe priorly by its differential equations. 

Afterwards, the design is followed by the discussion of the the similar modern control methods but without extending the states with the disturbance. In order to make the design closer to the real-life implementation, the chapter continues on describing the control in both continuous and discrete time. 

By the end of the chapter, simulation is carried out in Simulink in order to show the results of the control. 